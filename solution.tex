\documentclass[14pt,a4paper]{article}

\usepackage[UTF8]{ctex}
\usepackage[margin=1.8cm]{geometry}
\usepackage{amsmath,amssymb,booktabs,graphicx,hyperref,enumitem,fancyhdr,listings,float,minted,bookmark}
\usepackage{titlesec}
\usepackage[most]{tcolorbox}
\titleformat{\section}{\centering\LARGE\bfseries}{\thesection}{1em}{}
\pagestyle{fancy}
\fancyhf{}
\fancyhead[L]{\textbf{2025 - 1024 PDCC}}
\fancyhead[R]{\bf Page \thepage}
\renewcommand{\headrulewidth}{0.4pt}

\setlength{\parindent}{0pt}
\setlength{\parskip}{0.5em}
\linespread{1.2}
\setlist[itemize]{itemsep=0pt, topsep=2pt}
\setlist[enumerate]{itemsep=0pt, topsep=2pt}


\lstset{
  basicstyle=\ttfamily\small,
  breaklines=true,
  frame=single,
  numbers=left,
  numberstyle=\tiny,
  backgroundcolor=\color[gray]{0.96},
  xleftmargin=1.5em,
}
\newtcolorbox{hintbox}{
  enhanced,
  colback=blue!5!white, 
  colframe=blue!50!black,
  fonttitle=\bfseries,      % 标题加粗
  coltitle=white,           % 标题颜色
  title=Hint,               % 框标题
  sharp corners,
  boxrule=0.5pt,
  left=6pt, right=6pt, top=6pt, bottom=6pt,
  before skip=10pt, after skip=10pt,
  breakable,                % 允许跨页
}

\begin{document}

\vspace*{\fill}
\begin{center}
  {\Huge \bfseries 1024 Coding Challenge (Algorithmic)}\\[1em]
  { \Large 2025 Programmer's Day }\\[1em]
  { \Large Full Solution}\\[1em]
\end{center}
\vspace*{\fill}
\clearpage

\section{台风}
签到题,输出 \texttt{Reject!} 即可。

\section{图书馆整理计划}
签到题,输出 $\left(\sum_{i=1}^{n} a_i\right)/B\times X + \left(\sum_{i=1}^{n} a_i\right) \times Y \bmod B $ 即可。


\section{星辉廊道}
显然,本题可以被简单的化简为\textbf{最大化路径上所有边权的按位与}。

\begin{hintbox}
考虑按位与运算的性质:若结果的某一位为 $1$,则路径上所有边该位都必须为 $1$。
\end{hintbox}

我们可以采用按位枚举的套路确定答案。

具体而言,我们从高位往低位逐步构造答案 $\text{ans}$:

\begin{enumerate}
  \item 初始 $\text{ans}=0$。
  \item 从第 $29$ 位到第 $0$ 位,对于每一位 $b$):
  \begin{itemize}
  \item 令 $\text{test} \leftarrow \text{ans} \lor (1 \ll b)$,假设当前位为 $1$。
  \item 仅保留所有满足 $(w \land \text{test}) = \text{test}$ 的边(即选择所有在第 $b$ 位上是 $1$ 的边)。
  \item 在这个新图中暴力 BFS 检查 $1$ 是否能到达 $N$。
  \item 若能到达,则说明该位可以保留,更新答案;否则该位不能保留,保持答案不变。
  \end{itemize}
\end{enumerate}

\begin{minted}{cpp}
signed main(){
    cin.tie(nullptr)->sync_with_stdio(0);
    cin>>n>>m;
    for(int i=1;i<=m;i++){
        int u,v,w;cin>>u>>v>>w;
        add(u,v,w),add(v,u,w);
    }
    auto chk=[&](int flag){
        vector<int> vis(n+1,0);
        queue<int> q; vis[1]=1; q.push(1);
        while(!q.empty()){
            int u=q.front();q.pop();
            for(int i=head[u];i;i=edge[i].next){
                int v=edge[i].to, w=edge[i].w;
                if(!vis[v] && ((w&flag)==flag)){
                    vis[v]=1;
                    q.push(v);
                }
            }
        }
        return vis[n];
    };
    for(int s=29;s>=0;s--){
        int tmp=cnt|(1<<s);
        if(chk(tmp)) cnt=tmp;
    }
    cout<<cnt;
    return 0;
}
\end{minted}

\section{File Restoration}
又是一道小清新 DP 好题!Allen 太喜欢 DP 了(什

观察题面,一个很好的思路是将每一个文件夹抽象为一个节点,则整个文件夹目录可以被视作一棵树。

考虑如何设计状态以及如何进行树上转移。

由于两种操作一种是单点翻转一种是子树翻转,我们从相对来说较难的子树翻转入手。

\paragraph{状态}
子树翻转每次都会将一个父节点的所有子节点以及其本身直接翻转,因此 DP 状态中肯定需要一个东西去记录和转移继承下来的翻转信息。

设 $p\in\{0,1\}$ 为到达当前节点时,因父节点的 “子树翻转” 累积到它身上的翻转奇偶。

$dp_{u,p}$ 就可以被设计为:

\textbf{在节点 $u$,在已承受奇偶为 $p$ 的情况下,让 $u$ 的整棵子树全为 $1$ 的最小操作数。}

\paragraph{转移}
有了状态,转移就是 trivial 的了。

我们仅需在 $u$ 处决定是否做一次“以 $u$ 为根的子树翻转”。

记该决策 $y\in\{0,1\}$,做了之后传下去的懒标记变为 $p' = p\oplus y$。

$v$ 的当前值变成 $a_v\oplus p'$。若为 $0$,需要再做一次 “单点翻转 $v$”。

子树代价显然是 $\sum_{\forall \text{son}} dp_{\text{son},p'}$。

\begin{minted}{cpp}
int arr[maxn];
vector<int> e[maxn];
int dp[maxn][2];

void dfs(int u, int fa){
    for(int v:e[u]){
        if(v!=fa) dfs(v,u);
    }
    for(int p=0;p<=1;p++){
        int best=1e9+7;
        for(int y=0;y<=1;y++){
            int flag=arr[u]^p^y^1;
            int cost=y+flag;
            for (int v:e[u]){
                if(v!=fa) cost+=dp[v][p^y];
            }
            best=min(best,cost);
        }
        dp[u][p]=best;
    }
}

signed main(){
    cin.tie(nullptr)->sync_with_stdio(0);
    cin>>n;
    for(int i=1;i<=n;i++) cin>>arr[i];
    for(int i=1;i<=n-1;i++){
        int u,v;cin>>u>>v;
        e[u].pb(v);e[v].pb(u);
    }
    dfs(1,0);
    cout<<dp[1][0]<<endl;
    return 0;
}
\end{minted}

\section{抽卡打怪}
来一发验题人题解。

\section{「星光渐明之时」}

\end{document}
